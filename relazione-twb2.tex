\documentclass[12pt]{article}
\usepackage{hyperref}
\usepackage{graphicx}
\usepackage[font=small,labelfont=bf]{caption}
\title{Analisi sito}
\date{}
\author{carlo Sindico}
\begin{document}
\pagenumbering{arabic}



\begin{titlepage}

\newcommand{\HRule}{\rule{\linewidth}{0.5mm}} % Defines a new command for the horizontal lines, change thickness here

\center % Center everything on the page
 
%----------------------------------------------------------------------------------------
%	HEADING SECTIONS
%----------------------------------------------------------------------------------------

\textsc{\LARGE Universit\`a degli Studi di Padova}\\[1.5cm] % Name of your university/college
\textsc{\Large Laurea in Informatica}\\[0.5cm] % Major heading such as course name
\textsc{\large Corso di Tecnlogie Web 2}\\[0.5cm] % Minor heading such as course title

%----------------------------------------------------------------------------------------
%	TITLE SECTION
%----------------------------------------------------------------------------------------

\HRule \\[0.4cm]
{ \huge  Progetto di fine corso}\\[0.3cm] % Title of your document
\HRule \\[1.5cm]
 
%----------------------------------------------------------------------------------------
%	AUTHOR SECTION
%----------------------------------------------------------------------------------------

\begin{minipage}{0.4\textwidth}
\begin{flushleft} \large
\emph{Studente:}\\
Carlo \textsc{Sindico} % Your name
\end{flushleft}
\end{minipage}
~
\begin{minipage}{0.4\textwidth}
\begin{flushright} \large
\emph{Matricola:} \\
\textsc{1069322} % Supervisor's Name
\end{flushright}
\end{minipage}\\[4cm]

% If you don't want a supervisor, uncomment the two lines below and remove the section above
%\Large \emph{Author:}\\
%John \textsc{Smith}\\[3cm] % Your name

%----------------------------------------------------------------------------------------
%	DATE SECTION
%----------------------------------------------------------------------------------------

{\large 12/06/2016}\\[3cm] % Date, change the \today to a set date if you want to be precise

%----------------------------------------------------------------------------------------
%	LOGO SECTION
%----------------------------------------------------------------------------------------

%\includegraphics{Logo}\\[1cm] % Include a department/university logo - this will require the graphicx package
 
%----------------------------------------------------------------------------------------

\vfill % Fill the rest of the page with whitespace

\end{titlepage}
\newpage
\renewcommand{\contentsname}{Indice}
\tableofcontents

\newpage
\pagenumbering{arabic}

\section{Una breve introduzione}
\subsection{Liceo Leopardi Majorana}
\begin{itemize}
	\item Il sito fornisce uno spazio web per le scuole superiori Liceo Leopardi Majorana. Il sito \'e\ destinatinato a diverse categorie di persone:
	\item per tutte le persone che sono interessate a visitare la scuola, il sito offre una sezione apposita,su come raggiungere la scuola, e anche la storia del liceo.
	\item gli studenti hanno la possibilit\'a\ di visualizzare i propri voti e la media scolastica(accedendo al registro elettronico) e di accedere ad una piattaforma moodle, a cui possono iscriversi,dove sono presenti tutti gli insegnamenti; oltre a visualizzare le news su attivit\'a\ o progetti che la scuola organizza.il sito offre agli studenti una sezione dedicata alla biblioteca presente all'interno della scuola, con un opportuno catalogo (anche se in una versione minimale) dei libri presenti in essa. altre funzionalità offerte agli studenti sono la visualizzazione del calendario e degli orari delle lezioni.
	\item gli insegnanti hanno la possibilit\'a\ di accedere a diverse sezioni d sito come le graduatorie di istituito(gli insegnanti visualizzano quali classi gli sono state assegnate). gli insegnanti inoltre gestiscono il registro elettronico.
\end{itemize}



\end{document}