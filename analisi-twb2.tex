
\documentclass[a4paper,11pt]{report}
\usepackage[T1]{fontenc}
\usepackage[italian,english]{babel}
\usepackage[utf8]{inputenc}
\usepackage[xindy]{imakeidx}
\usepackage{xcolor}
\usepackage{graphicx}
\usepackage{amsmath}
\usepackage{amssymb}
\usepackage{etaremune}
\usepackage{enumitem}
\usepackage[toc,page]{appendix}
\usepackage{verbatimbox}

\usepackage[hidelinks, colorlinks=true]{hyperref}	
\usepackage{bookmark}
\usepackage{caption}
\usepackage{subfig}

\captionsetup{tableposition=top, figureposition=bottom, font=small}
\setcounter{tocdepth}{3}
\setcounter{secnumdepth}{3}


\usepackage{epigraph}


\makeindex


\includeonly{
			sezioni/PrimiPassiNelWeb,%
			sezioni/IlSitoWeb,%
			sezioni/ProblemiDiUsabilita,%
			sezioni/SitiEcommerce,%
			sezioni/IlComportamentoDegliUtenti,%
			sezioni/LaPubblicita,%
			sezioni/LaRicerca,%
			sezioni/LaVisibilita,%
			sezioni/IlNomeGiusto,%
			sezioni/LInformazioneEIlWebSemantico,%
			sezioni/MobileWeb,%
			sezioni/SocialWeb,%
			appendici/AnalisiUsabilitaSito,
			appendici/CookieLaw,%
			%appendici/EyeTrackIII,%
			%appendici/WebSpamTaxonomy,%
			%appendici/LOD,%
			%appendici/ScrivereRelazioneUsabilita,%
			%appendici/AffrontareEsame%
			}		
		
\begin{document}

\title{Lezioni su Tecnologie Web 2}
\author{Eduard Bicego}
\date{14-04-2016}

\maketitle

\input{sezioni/Dedica}

\begin{abstract}
	Il presente documento ha l'intento di raccogliere tutto il materiale presentato al corso di tecnologie web 2 dell'università di Padova tenuto dal professor Massimo Marchiori. Il documento ha lo scopo di dare le informazioni necessari ad ulteriore ricerca e approfondimento, come del resto le lezioni del corso. Nonostante gli sforzi questo documento è lungi dall'essere una guida completa su usabilità, motori di ricerca, web semantico e altro. \par
	
	Ho cercato di raggruppare e strutturare nel modo più logico le informazioni al fine di rendere più facile l'apprendimento cercando di mantenere intatto lo stile e il contenuto come è spiegato in aula. \par
	
	Il seguente documento non si prefigge lo scopo di contenere lo stretto indispensabile per l'esame e tanto meno di contenerne tutto il necessario. 
	È usufruibile a tutti nella speranza di essere materiale utile e di supporto. Il simbolo asterisco tra parentesi tonde "(*)" significa che la sezione non è stata sviluppata sufficientemente. Qualsiasi \emph{fork} o incremento su tale documento è ben accetto purché rimangano i riferimenti a me e alle risorse citate in riferimenti.
	\begin{flushright}
		Buon viaggio virtuale \\
		\emph{Eduard}
	\end{flushright}
	
	
\end{abstract}

\hypersetup{linkcolor=black}
\tableofcontents
	%\newpage
\listoffigures
	%

\newpage
	\begin{center}
		\emph{``Abbiamo Internet, il web, gli smartphone,\\
		 ma chi li maneggia sono sempre gli stessi,\\
		  siamo sempre noi uomini, \\
		  con le nostre luci e le nostre ombre \\
		  che sfidano ogni tempo e ogni epoca.''\\}
	\end{center}
	\begin{flushright}
		\emph{Massimo Marchiori} - \emph{Meno Internet più Cabernet}	
	\end{flushright}

\newpage

\hypersetup{linkcolor=blue, urlcolor=blue}

\include{sezioni/PrimiPassiNelWeb}

\include{sezioni/IlSitoWeb}

\include{sezioni/ProblemiDiUsabilita}

\include{sezioni/SitiEcommerce}

\include{sezioni/IlComportamentoDegliUtenti}

\include{sezioni/LaPubblicita}

\include{sezioni/LaRicerca}

\include{sezioni/LaVisibilita}

\include{sezioni/IlNomeGiusto}

\include{sezioni/LInformazioneEIlWebSemantico}

\include{sezioni/MobileWeb}

\include{sezioni/SocialWeb}


\begin{appendices}

\include{appendici/AnalisiUsabilitaSito}

\include{appendici/CookieLaw}

%\include{sezioni/}

%\include{sezioni/}

\end{appendices}

%\newpage
	\input{biblio/biblio}

\end{document}	
